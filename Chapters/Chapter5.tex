
\chapter{Conclusion} % Main chapter title

\label{Chapter5} %for referencing this chapter elsewhere, use \ref{ChapterX}

\lhead{\emph{Conclusion}} %this is for the header on each page - perhaps a shortened title



A national review of the status of computer science K-12 education spawns a variety of questions that must be addressed as a CS program develops at Newman. In the Lower School, those questions relate to the balance of coding vs. computer literacy, classroom vs. Media time, physical vs. virtual instruction, and game-based vs. open-ended creative platforms. In the Middle School, which is less amenable to additional electives, the question is: should CS be integrated into the classroom or offered as a standalone course? How will we make space for a new ``core'' subject in an already tight schedule? And for the Upper School, the questions relate to curriculum---should Newman offer an academic approach to CS akin to an introductory course at the university level, or should an upper school program place greater emphasis on the creative, innovative, and applied coding process from industry and code schools? What are the merits of offering the AP? And finally, can online CS instruction replace classroom instruction? \par
While many of these questions can be debated, the dire need for computer science at Newman cannot. Students are graduating without valuable, marketable, and creative computational thinking skills. If Newman is to remain a competitive, top-notch independent school, it must keep pace with this evolving field.\par
Fortunately, even the independent schools with standout programs---schools like Winchester Thurston and Greenhill---are relatively new to CS. Using nationally established benchmarks (CSTA, ISTE, AP, etc.), learning from the experiences of top tier schools and institutions, and leveraging new tools and technologies, Newman has the ability to craft a holistic K-12 CS plan and establish itself as a local and national leader in computer science education.\par
