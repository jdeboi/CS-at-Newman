% Chapter Template

\chapter{Recommendations} % Main chapter title

\label{Chapter3} % Change X to a consecutive number; for referencing this chapter elsewhere, use \ref{ChapterX}

\lhead{Chapter 3. \emph{Recommendations}} % Change X to a consecutive number; this is for the header on each page - perhaps a shortened title


%----------------------------------------------------------------------------------------
\section{Lower School}
\section{Middle School}
Possible implementation strategies to explore: CS inegrated in the classroom, 6th grade science course, club time, flex time, new schedule

\section{Upper School}
Unlike the Middle School schedule, the Upper School sequence affords time and space for computer science electives, as well as a graduation requirement. The rotating eight block schedule means there are 32 possible credits, although Newman students rarely graduate with greater than 30, typically averaging approximately 27 credits. The student handbook stipulates that students must complete 24 academic credits to graduate, taken from the following subject areas: \par

\begin{table}
	\begin{center}
\begin{tabular}{ | l | c | }
	
	\hline
	\textbf{Subject} & \textbf{Credits} \\ \hline
	English & 4 \\ \hline
	Mathematics & 4 \\ \hline
	History/Social Studies & 4 \\ \hline
	Science & 3 \\ \hline
	World Languages & 3 \\ \hline
	Arts & 2 \\ \hline
	Physical Education & 2 \\ \hline
	Electives & 2 \\ \hline
	\textbf{Total} & \textbf{24} \\ \hline
\end{tabular} 
\caption{Upper School Graduation Requirements} \label{tab:usreqs} 
\end{center}
\end{table}
\par
The ``elective" graduation requirement is the most obvious space to insert CS courses. It's worth examining which departments and classes may be impacted by these additional electives.\par
If the ultimate objective is to treat CS as foundational subject on par with core subjects like math and English, a critical analysis of the Upper School curriculum will be required to make space for a new set of mandated credits. Given the nascent status of computer science at Newman, this document will not attempt to evaluate the merits of the existing sequence or propose potentially subversive structural changes. Nevertheless, it's instructive to examine schools with robust CS requirements and evaluate how a similar program at Newman might affect existing courses.\par


\hiddensubsection{Standards}

\hiddensubsection{AP Computer Science}

\hiddensubsection{Online Education}

\hiddensubsection{Tiered CS Electives}
A scaffolded series of CS courses is the best way to ensure 

\hiddensubsection{CS Department} 
\hiddensubsection{Innovation, Design Thinking, \& CS}
\hiddensubsection{CS Graduation Requirement} \par
This document strongly recommends that the Upper School adopt a 1/2 credit CS graduation requirement. A mandated credit, as opposed to an optional elective, ensures that every student- especially groups typically underrepresented in CS- get exposure to coding prior to college and have the opportunity to explore coding career possibilities prior to choosing a college major or minor. \par

In addition to introducing foundational computational/coding concepts (explored in the ``Benchmarks" section), this introductory CS course should prioritize the following the exploration diverse career opportunities afforded by CS/computational thinking, as well as emphasize the potential for innovation and creative expression through code. Assuming this 1/2 credit course is the single point of CS exposure, the course needs to serve as the hook. Given the multitude of coding languages and platforms (refer to Appendix \ref{AppendixTools}), a highly-engaging project-based course that explores a myriad of hardware and software tools is \textit{both feasible and critically important} for an effective introductory CS experience. \par
Included in Appendix \ref{AppendixIntro} are suggested syllabi for two 1/2 credit courses-- ``Creative Coding" and ``Physical Computing." Each course is designed to meet the learning objectives of an introductory series as outlined above and as specified by CSTA Benchmarks. \par
- staffing requirements \par
- space requirements \par
- potential for online, self-guided instruction \par