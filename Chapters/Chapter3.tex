% Chapter Template

\chapter{Recommendations} % Main chapter title

\label{Chapter3} % Change X to a consecutive number; for referencing this chapter elsewhere, use \ref{ChapterX}

\lhead{Chapter 3. \emph{Recommendations}} % Change X to a consecutive number; this is for the header on each page - perhaps a shortened title


%----------------------------------------------------------------------------------------
\section{Lower School}
\section{Middle School}
Possible implementation strategies to explore: CS integrated in the classroom, 6th grade science course, club time, flex time, new schedule

\section{Upper School}
Unlike the Middle School schedule, the Upper School sequence affords time and space for computer science electives, and potentially a graduation requirement. The rotating eight block schedule means there are 32 possible credits, although Newman students rarely graduate with greater than 30, typically averaging approximately 27 credits. The student handbook stipulates that students must complete 24 academic credits to graduate, taken from the following subject areas: \par

\begin{table}
	\begin{center}
\begin{tabular}{ | l | c | }
	
	\hline
	\textbf{Subject} & \textbf{Credits} \\ \hline
	English & 4 \\ \hline
	Mathematics & 4 \\ \hline
	History/Social Studies & 4 \\ \hline
	Science & 3 \\ \hline
	World Languages & 3 \\ \hline
	Arts & 2 \\ \hline
	Physical Education & 2 \\ \hline
	Electives & 2 \\ \hline
	\textbf{Total} & \textbf{24} \\ \hline
\end{tabular} 
\caption{Upper School Graduation Requirements} \label{tab:usreqs} 
\end{center}
\end{table}
\par
The ``elective" graduation requirement is the most obvious space to insert CS courses. It's worth examining which departments and classes may be impacted by these additional electives.\par
If the ultimate objective is to treat CS as foundational subject on par with core subjects like math and English, a critical analysis of the Upper School curriculum will be required to make space for a new set of mandated credits. Given the nascent status of computer science at Newman, this document will not attempt to evaluate the merits of the existing sequence or propose potentially subversive structural changes. Nevertheless, it's instructive to examine schools with robust CS requirements and evaluate how a similar program at Newman might affect existing courses.\par


\hiddensubsection{Standards}

\hiddensubsection{AP Computer Science}
For a full analysis of the pros and cons of AP Computer Science, please refer to %TODO. 
This document strongly recommends against offering AP Computer Science A, especially as a first entry into CS, because of the curriculum's rigidity, lack of creativity, and use of the Java programming language; however, for a select group of students, typically those with existing CS experience and a strong committment to the subject, an independent study or online course like CodeHS may be advisable. This AP does not have broad appeal and is not ideal for inspiring students; it is, however, both rigorous and aligned to many college-level courses. \par
AP Computer Science Principles, on the other hand, may offer both the national recognition of an AP while still preserving the creative and innovative component of CS. This new AP is young, so time and experimentation may be necessary to determine the course's true value. Offering AP CS Principles (or an equivalent but unaccredited course that prepares students for success on the AP) as a second or third tier, year-long course is recommended.\par


\hiddensubsection{Online Education}
For a full analysis of online platforms, their merits and shortcomings, please refer to %TODO.
There are several examples where online programs add clear value to Newman's CS offerings. Allowing platforms like Global Online Academy or CodeHS to fill niche needs- such as allowing self-directed students to take specific upper level courses- is cost-effective and grants students more options. In addition, using Kahn Academy or other online resources as a crutch or supplement may bolster in-class instruction.\par
This document, however, does not recommend the strict reliance on online CS platforms, especially for meeting a potential graduation requirement. By the very nature of being ``cookie cutter'' and purely virtual, online platforms cannot offer the same ``rigor and relationships'' as curriculum spearheaded by a teacher. Hooking students, inspiring students, communicating the importance and broad applicability of CS, and ensuring that students recognize the creative and innovative potential of coding- these must be the priorities of a CS program at Newman. Without the enthusiasm and relationships created by and with teachers, the personally-relevant and curated projects, or the hands-on communication, students will be less likely to pursue or take interest in CS.\par 
To summarize: the human element is the value added by independent school education. Newman is known for its top-notch pedagogy; CS, a critically important field in the 21st century, cannot be the exception. \par
All of that said, hooking students is the priority, after which students may want more freedom to explore advanced topics on their own. Using online CS programs as a second or third tier course may be advisable, but only if resources or hiring difficulties prohibit an in-person CS experience. \par   

\hiddensubsection{Stage 1}


\hiddensubsection{Stage 2} 
\hiddensubsection{Stage 3}
\hiddensubsection{CS Graduation Requirement} \par
This document strongly recommends that the Upper School adopts a 1/2 credit CS graduation requirement. A mandated credit, as opposed to an optional elective, ensures that every student- especially groups typically underrepresented in CS- get exposure to coding prior to college. \par


- staffing requirements \par
- space requirements \par