% Chapter Template

\chapter{Recommendations} % Main chapter title

\label{Chapter3} % Change X to a consecutive number; for referencing this chapter elsewhere, use \ref{ChapterX}

\lhead{Chapter 3. \emph{Recommendations}} % Change X to a consecutive number; this is for the header on each page - perhaps a shortened title


%----------------------------------------------------------------------------------------
\section{Lower School}
\section{Middle School}
Possible implementation strategies:
\begin{enumerate}
	\item CS inegrated in the classroom
	\item 6th grade science course
	\item club time
	\item flex time
	\item new schedule
\end{enumerate}

\section{Upper School}
Unlike the Middle School schedule, the Upper School sequence affords time and space for computer science electives, as well as a graduation requirement. The rotating eight block schedule means there are 32 possible credits, although Newman students rarely graduate with greater than [INSERT], typically averaging approximately [INSERT] credits. The student handbook stipulates that students must complete 24 academic credits to graduate, taken from the following subject areas: \par

\begin{table}
	\begin{center}
\begin{tabular}{ | l | c | }
	
	\hline
	\textbf{Subject} & \textbf{Credits} \\ \hline
	English & 4 \\ \hline
	Mathematics & 4 \\ \hline
	History/Social Studies & 4 \\ \hline
	Science & 3 \\ \hline
	World Languages & 3 \\ \hline
	Arts & 2 \\ \hline
	Physical Education & 2 \\ \hline
	Electives & 2 \\ \hline
	\textbf{Total} & \textbf{24} \\ \hline
\end{tabular} 
\caption{Upper School Graduation Requirements} \label{tab:usreqs} 
\end{center}
\end{table}
\par
The "elective" graduation requirement is the most obvious space to insert CS courses. It's worth examining which departments and classes may be impacted by these additional electives. [INSERT] these are the types of electives currently offered , number of students enrolled. Are art classes going to suffer? Do they make up the bulk of the extra electives? \par
what will senior year look like if you opt to take AP CS?\par
If the ultimate objective is to treat CS as foundational subject on par with core subjects like math and English, a critical analysis of the Upper School curriculum will be required to make space for a new set of mandated credits. Given the nascent status of computer science at Newman, this document will not attempt to evaluate the merits of the existing sequence or propose potentially subversive structural changes. It is, however, instructive to examine schools with robust CS requirements, .\par
  [insert schools with lots of cs] \par 

\subsection{Stage 1}
This document recommends, at a bare minimum, that the Upper School adopt a 1/2 credit CS graduation requirement. A mandated credit, as opposed to an optional elective, ensures that every student- especially groups typically underrepresented in CS- get exposure to coding prior to college and have the opportunity to explore coding career possibilities prior to choosing a college major or minor. \par
In addition to introducing foundational computational/coding concepts (explored in the ``Benchmarks" section), this introductory CS course should prioritize the following objectives: \par 
\begin{enumerate}
	\item Explore diverse career opportunities afforded by CS/computational thinking
	\item Emphasize the potential for innovation and creative expression through code
\end{enumerate} \par
Assuming this 1/2 credit course is the single point of CS exposure, the course needs to serve as the hook. There are infinite possibilities for highly-engaging, interactive coding experiences (see the Appendix [insert] for a table of tools and platforms), interactive Javascript visualizations, and so much more. A fun project-based course that explores a myriad of hardware and software tools is \textit{both feasible and critically important} for an effective introductory CS experience. \par
Included in the Appendix are suggested curricula for two 1/2 credit courses designed to introduce CS from an art and engineering perspective respectively: ``Creative Coding" and ``Physical Computing." Each course is designed to meet the learning objectives of an introductory series as outlined above, specified by CSTA Benchmarks, and necessary to introduce AP CS topics. \par
[what other courses are we referencing insert] \par
- Staffing requirements
would not do online for a graduation requirement; what's out there isn't engaging enough; doesn't afford exploration or convey the diversity of subjects possible with coding
how many sections 
- space requirements
\subsection{Stage 2}
To develop a robust series of CS electives, 
1/2 credit before CS principles? 
AP CS Principles

\subsection{CS as a Core Subject}
\subsection{Innovation, Design Thinking, \& CS}
[insert 60\% of jobs don't exist]

barriers?
- teachers
staffing
Number of CS requirements * number of students per grade / average class size = number of sections
number of sections / 5 = number of faculty