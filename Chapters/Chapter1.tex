% Chapter 1

\chapter{Research} % Main chapter title

\label{Chapter1} % For referencing the chapter elsewhere, use \ref{Chapter1} 

\lhead{Chapter 1. \emph{Research}} % This is for the header on each page - perhaps a shortened title

%----------------------------------------------------------------------------------------

\section{Case Studies}
\subsection{Index Schools}
\subsection{Other Independent Schools}
Phillips Exeter Academy Exeter, NH. \\
Ransom Everglades School Coconut Grove, FL. \\
The College Preparatory School Oakland, CA. \\
Horace Mann School Bronx, NY. \\
Castilleja School Palo Alto, CA. \\
Trinity School New York, NY. \\
The Hotchkiss School Lakeville, CT. \\
San Francisco University High School San Francisco, CA. \\
Dalton School \\
\subsection{Public Schools}
\subsection{Code Schools}
\subsection{Higher Education}
%----------------------------------------------------------------------------------------

\section{National Standards}
\subsection{CSTA}
The Computer Science Teachers Association (CSTA) released the revised "K-12 Computer Science Standards" in 2011. These standards can be found in Appendix \ref{AppendixCSTA}.
\subsection{ISTE}
The International Society for Technology in Education (ISTE) has compiled a set of standards for computer science educators. These standards are listed in Appendix \ref{AppendixISTE}. 
\subsection{AP}
Since 2003 the \href{http://media.collegeboard.com/digitalServices/pdf/ap/ap-course-overviews/ap-computer-science-a-course-overview.pdf}{Advanced Placement Computer Science A} exam has tested students' ability to solve problems and design algorithms using Java. The College Board's test is intended to cover the equivalent of a first semester college computer science course. The topics include:
\begin{enumerate}
	\item Object-Oriented Program Design:  Program and class design 
	\item Program Implementation: Implementation techniques. Programming constructs, Java library classes
	\item Program Analysis: Testing, Debugging, Runtime exceptions, Program correctness, Algorithm analysis
	\item Standard Data Structures:  Primitive data types, Strings, Classes, Lists, Arrays (1-D and 2-D) 
	\item Standard Operations and Algorithms: Searching, Sorting
	\item Computing in Context: System reliability, Privacy, Legal issues and intellectual property, Social and ethical ramifications of computer use 
\end{enumerate}
\newenvironment{blockquote}{%
  \par%
  \medskip
  \leftskip=4em\rightskip=4em%
  \noindent\ignorespaces}{%
  \par\medskip}
In 2016 the College Board plans to release a second AP CS course, \href{https://secure-media.collegeboard.org/digitalServices/pdf/ap/ap-computer-science-principles-curriculum-framework.pdf}{AP Computer Science Principles}, that will operate in tandem to the other AP CS class. From the AP website, ``AP Computer Science Principles offers a multidisciplinary approach to teaching the underlying principles of computation." Unlike the AP CS A which relies exclusively on Java, the Principles course allows teachers to use any programming language. The course introduces key programming concepts with a ``focus on fostering students to be creative." The ``Big Ideas" covered by this exam include:
\begin{enumerate}
	\item Creativity
	\item Abstraction
	\item Data and Information
	\item Algorithms
	\item Programming
	\item Internet
	\item Global Impact
\end{enumerate} \par
\begin{table}
	\centering
	\bgroup
	\def\arraystretch{1.5}
	\begin{tabular}{ p{7cm} p{7cm} }
		\hline
		\textbf{AP Computer Science \textit{A}} &	\textbf{AP Computer Science \textit{Principles}} \\ \hline \hline
		\multicolumn{2}{l}{\textit{Curriculum:}} \\
		Focused on object-oriented programming and problem solving & Built around fundamentals of computing including problem solving, working with data, understanding the Internet, cybersecurity, and programming. \\ \hline
		\multicolumn{2}{l}{\textit{Language:}} \\
		Java is the designated programming language	& Teachers choose the programming language(s) \\ \hline
		\multicolumn{2}{l}{\textit{Skillset:}} \\
		Encourages skill development among students considering a career in computer science or other STEM fields & Encourages a broader participation in the study of computer science and other STEM fields, including AP Computer Science A \\ \hline
		\multicolumn{2}{l}{\textit{Assessment:}} \\
		Multiple-choice and free-response questions (written exam) & Two performance tasks students complete during the course to demonstrate the skills they have developed (administered by the teacher; students submit digital artifacts), and multiple-choice questions (written exam) \\ \hline
	\end{tabular}
	\egroup 
	\caption{Comparison of AP Computer Science Exams ~\cite{APcomp}} \label{tab:apexams} 
\end{table}\par
For a detailed list of learning benchmarks associated with this exam, \href{https://secure-media.collegeboard.org/digitalServices/pdf/ap/ap-computer-science-principles-curriculum-framework.pdf}{check out the AP Computer Science Principles Curriculum Framework}.
\subsection{Next Generation Science Standards}
From Next Generation Science Standards: 
\begin{blockquote}
``Computer Science and Statistics Computer science and statistics are other areas of science that are not addressed here, even though they have a valid presence in K-12 education... Computer science, too, can be seen as a branch of the mathematical sciences, as well as having some elements of engineering. But, again, because this area of the curriculum has a history and a teaching corps that are generally distinct from those of the sciences, the committee has not taken this domain as part of our charge. Once again, this omission should not be interpreted to mean that computer science or statistics should be excluded from the K-12 curriculum. There are aspects of computational and statistical thinking that must be understood and applied in learning about the sciences, and we identify these aspects, along with mathematical thinking, in our discussion of science practices in Chapter 3." 
\end{blockquote}\par
Although the NGSS does not expliciting outline CS benchmarks, the Engineering Design standards are highly applicable to software engineering or computational thinking projects. These benchmarks are covered in Appendix \ref{AppendixNGSS}.
\subsection{Common Core}
%----------------------------------------------------------------------------------------
\section{Online Tools}
\section{Hardware Tools}