% Chapter 1

\chapter{Introduction} % Main chapter title

\label{Introduction} % For referencing the chapter elsewhere, use \ref{Chapter1} 

\lhead{Introduction} % This is for the header on each page - perhaps a shortened title

%----------------------------------------------------------------------------------------

\section{What is computer science?}
For the purpose of avoiding potentially derailing semantics, it's instructive to begin with definitions of computer-related concepts used throughout this document. \par
\textbf{``Coding"} is the latest term to gain momentum in the education vernacular, popularized by the \href{https://hourofcode.com/us}{Hour of Code} and online learning platforms like Scratch. While ``coding" may be used to convey the beginning steps of \textbf{computer programming} \cite{huffpo}, the terms are used synonymously and can be defined as follows: the process of writing or compiling machine instructions to accomplish a task. Coding is most commonly associated with the creation of websites, software applications, and mobile apps; however, there are innumerable potential applications of code. There are languages to program music, wearable tech, autonomous drones, interactive art visualizations, financial data, research documents (like this one!), 3D printers, robotics, and the list goes on. \par
Perhaps less common but equally as important is the concept of \textbf{``computational thinking"} (CT). Jeanette Wing, the Director of the Carnegie Mellon Center for Computational Thinking, is arguably the projenitor of the term \cite{backtoschool}. The center describes CT as ``a way of solving problems, designing systems, and understanding human behavior that draws on concepts fundamental to computer science... [by] making use of different levels of abstraction and thinking algorithmically" \cite{cmct}. Google for Education defines CT as ``a problem solving process that includes a number of characteristics, such as logically ordering and analyzing data and creating solutions using a series of ordered steps (or algorithms), and dispositions, such as the ability to confidently deal with complexity and open-ended problems" \cite{googed}. To summarize, CT is a set of problem-solving mindsets independent of computer programming that are applicable to all disciplines, not just computing. Emphasis on CT, especially in early K-12 education, will lay a strong foundation for future coursework in a myriad of fields.\par

\textbf{Computer science} (CS) employs computational thinking mindsets in the study and execution of computer programs and algorithmic processes. Coding is an essential, but not exclusive, component of CS, as is the study of hardware and software designs, their applications, and their impact on society. \par
While coding, computational thinking, and computer science are used interchangably throughout this document to refer to a set of K-12 curriculum encompassing all three, is important to distinguish ``computer science'' from \textbf{``computer literacy.''} The National Center for Women and Information Technology states, ``the ability to create and adapt new technologies distinguishes computer science from computer literacy, which focuses more on using existing technologies (e.g., word processing, spreadsheets)" \cite{wit}. Computer science teaches students how to think logically, solve hard problems, and understand the internal processes of computing rather than merely understanding how to use specific software. Although this document will focus primarily on CS, computer literacy is neither obsolete nor irrelevant (e.g. keyboarding) in 21st century education. \par  

%----------------------------------------------------------------------------------------

\section{The importance of coding in K-12 education}
In September 2015, New York City Mayor Bill de Blasio announced that all of the city’s public schools would be required to offer computer science within the next 10 years. The announcement was neither the first of its kind nor the boldest; in December 2013, Chicago Mayor Rahm Emanuel launched Computer Science 4 All, a plan mandating a year-long computer science high school graduation requirement by 2019 \cite{chicagoCS}. And on the national stage, in early December 2015 President Obama signed into law a bipartisan bill - the Every Student Suceeds Act - recognizing computer science as a foundational academic subject on equal footing with subjects like math and English \cite{WSJref}. Municipal, state, and federal policymakers' recent focus on computer science education is no surprise. Education, especially K-12 education, is failing to keep pace with one of the most powerful tools of our time, jeopardizing the success and employment potential of American students.\par

\textbf{A NEW LITERACY} \\
In the rush by policymakers and educators to bring computer science to K-12 education, the term ``coding literacy'' has risen in the educational vernacular. While there are strong arguments for treating computer science as a ``core'' subject that permeates all levels of primary and secondary education, it's important to dissect the buzzword and extract the true foundational skills relevant to all students. \par
In a post by Chris Granger titled, ``Coding is Not the New Literacy,'' Granger equates coding to pens and pencils - tools, not fundamental skills, employed in the expression of a literacy. Just as reading comprehension and writing composition - processes of solidifying cogent thoughts - form the true basis of literacy, computational thinking mindsets and ``modeling," Granger argues, form the basis of this ``new literacy." He defines modeling as, ``creating a representation of a system (or process) that can be explored or used." \par 
In an age where programming languages and associated syntaxes, libraries, APIs, frameworks, hardware, and databases are rapidly-evolving, the true value of a computer science education lies not in an understanding coding mechanics or ``computer literacy'' (as defined in the previous section). It is the ability to think computationally and algorithmically, to develop and model systems, to decompose and debug complex problems, to explore and verify hypotheses rapidly and virtually: for these reasons, computer science is paramount in education. \par
Computer science happens to be an extremely powerful vector for modeling in almost any imaginable discipline. Just to list a few examples: modeling weather and climate patterns to predict the impacts of climate change, evaluating stress and strain of engineered structures, developing 3D animations for video games and entertainment, using image recognition algorithms to better detect breast cancer, creating financial models to measure the volatility in markets, or even, as researchers at the University of Maryland have shown, predicting the time and location of insurgent attacks in the Middle East\cite{insurgent}. \par
Computer science understood as modeling gives teachers a way to make every subject hands-on and concrete. Grady Simon, a Project Manager at Microsoft writes, 
\begin{blockquote}
	English classes teach models of English grammar, models of what it means to write well, models of narratives and arguments, and then they have students practice using those models to understand and write the English language. Computer science is the domain of pure models. The grammar of English sentences can be modeled as trees [a computer science data structure]. Arguments also tend to have a tree structure. Computer science talks about the properties of trees and grammars in general, among many other things. \par
	I think this is even more obvious in more technically rigorous subjects like science. Every scientific theory is a model, and because those models tend to be precise and rigorously defined, usually they lend themselves well to being instantiated as computer programs. Imagine a physics class where each homework assignment had students add a feature to a physics simulator, implementing in code the rules of physics they just learned about in class. Implement the laws of objects moving through space under gravitational field. Then you can have students play with the models. What happens if the gravitational constant were negative? What would it actually look like if gravity just imparted constant velocity downward instead of constant acceleration? 
	\end{blockquote} 
No matter what field of study students choose to persue, a solid foundation in computer science offers powerful ways to model, understand, and engage with hard problems. \par

\textbf{CREATIVITY AND INNOVATION} \\
To ignore the creative possibilities of coding is to ignore one of the most robust mediums of creative expression currently available. There are a myriad of programming tools for creating and communicating both physical and virtual ideas. Processing, a visual programming language designed for artists, is increasingly used to introduce programming thanks to its easy-to-learn platform and emphasis on the creation of art. ``Generative'' or autonomous virtual kaleidoscopes, interactive data visualizations, or colorful animated stories are just a few possibilities with an introductory-level knowledge of coding. \par
Max 7 is a block-based programming language for programming music, videos, and physical media. Innovative interfaces for synthesizing new sounds, video projections triggered by a dancer's body movements, and interactive LED light shows controlled by live music are just a few of the infinite possibilities. \par
Video game design and 3D animation enables artists to code worlds not bound by the contraints of reality. As virtual reality gains momentum thanks to cheap interfaces like Google Cardboard, there will be even greater opportunities for the creation of 3D programmed worlds. \par  
But the power of creative coding doesn't fall strictly into the domain of art. Research has shown that some of \textbf{the most effective learning experiences come when we make or design things}, especially those that have relevance to our lives or the lives of people around us \cite{papert1}. Computers can be viewed as a ``universal construction material'' granting rich opportunities for teaching or learning a diversity of subjects \cite{res98}. \par
Mitchel Resnick, the director of the Lifelong Kindergarten group at the Massachusetts Institute of Technology's Media Lab, is an expert on the use of coding as a creative medium for early education. He led the development of Scratch, a block-based coding platform that allows kids to create and share their own interactive stories, games, and animations. Resnick writes, ``Coding is not a set of technical skills but a new type of literacy and personal expression, valuable for everyone, much like learning to write. We see coding as a new way for people to organize, express, and share their ideas'' \cite{resonline}. \par
The creative potential of code has significant implications for entrepreneurial innovation as well as creative expression. In ``Rethinking Learning in the Digital Age,'' Resnick states:
\begin{blockquote} 
	The proliferation of digital technologies has accentuated the need for creative thinking in all aspects of our lives, and has also provided tools that can help us improve and reinvent ourselves. Throughout the world, computing and communications technologies are sparking a new entrepreneurial spirit, the creation of innovative products and services, and increased productivity. The importance of a well-educated, creative citizenry is greater than ever before \cite{resdig}.
\end{blockquote}
Coding is 21st century's universal tool for making, designing, and communicating. As technology rapidly evolves, so much our modes of innovating and creating. \par

\textbf{COMPUTER SCIENCE EMPLOYMENT} \\
Politicians on all levels of government are attempting to answer the technology industry's clarion call to fill a growing shortage of engineers.  \par
According to the U.S. Bureau of Labor Statistics, computer and information technology jobs will be some of the fastest growing occupations in America, making up approximately 2/3 of the 1.1 million new STEM jobs projected for 2012-2022. \cite{laborstats}. By 2022 the Bureau of Labor Statistics estimates there will be over 1 million open computing jobs \cite{laborstats}. Code.org, a non-profit dedicated to expanding computer science education, emphasizes that these computing jobs will fall under every industry, not just tech. Finance, business, government, manufacturing, and education are all industries projected to need computating skills in the next decade \cite{codeorgstats}.\par
Not only are unfilled computer science jobs abundant, they have high income potential. The highest paying degree in the U.S. is a computer science degree from Carnegie Mellon (\$84,000), and among all universities, the National Association of Colleges and Employers (NACE) survey ranks computer science as first or second in income potential \cite{forbeshigh}. \par
\textbf{WORKFORCE SHORTAGE} \\
Despite the high growth projections of well-paying computing jobs, skilled programmers aren't entering the workforce at a rate necessary to keep up with demand. Code.org estimates that there are 600,000 computing jobs open in the U.S. but only 38,175 students graduated with a computer science degree in 2013 \cite{codeorgstats}. 38,175 represents under 8\% of all STEM degrees earned in 2013, a problematic number considering that projections estimate 2/3 of all STEM jobs will be in a computing field \cite{laborstats}. \par
\textbf{K-12 EDUCATION GAP} \\
Education, especially K-12, has a significant role to play in the growing shortage of computer sciene graduates. Only 1 in 4 U.S. schools offer computer science classes with computer programming \cite{gallup}, and in 23 states computer science cannot be used to meet math or science graduation requirements \cite{WSJref}. According to the Gallup study \cite{gallup}, many school administrators do not perceive a demand by parents and students for computer science. The study states that the main reason school principals do not offer computer science is the lack of time for classes not required by testing, as well as the inability to hire computer science teachers due to availability and budget constraints \cite{gallup}. \par 
\textbf{UNDERREPRESENTED GROUPS} \\
Women and people of color are especially at risk of falling behind and losing access to high-paying computing employment. In 2011, women made up 21\% of students who took the AP computer science exam, and less than 1\% were African American \cite{backtoschool}. For many students, access to technology is the greatest hurdle. Hispanic students have less access to internet access at home than white or black students \cite{gallup}. \par
In the context of evaluating the importance of K-12 computer science education, it is critically important to point out that this opportunity gap begins in K-12 education. Table \ref{tab:csgap} indicates that by high school, women and people of color are underrepresented in the AP CS exam; this trend remains relatively stable through college and into the workforce. \par
\begin{table}
	\centering
		\begin{tabular}{ | l | p{3cm}|p{3cm}|p{3cm}| } \hline
			\textbf{Group} & \textbf{AP CS Exam} & \textbf{CS Bachelor’s} & \textbf{Computing Jobs} \\ \hline
			Females & 22\% & 17\% & 23\% \\ \hline
			People of Color & 13\% & 18\% & 14\% \\ \hline
		\end{tabular}
		\caption{Percentage of Underrepresented Groups in Computer Science \cite{codeorgstats}} \label{tab:csgap} 
\end{table} 
			
While closing the gender and achievement gap may seem daunting, early and frequent exposure to computer science throughout K-12 education has been demonstrated to have significant impacts on career outcomes. Women who try AP Computer Science in high school are ten times more likely to major in it in college, and Black and Hispanic students are seven times more likely \cite{apfive}. \par
\textbf{CONCLUSION} \\
Computing jobs are on the rise, and unless public policy and education quickly fall in line with the needs of an evolving workforce, American students will miss out on valuable opportunities to pursue high-paying careers in a myriad of fields. K-12 exposure to computer science, especially for women and people of color, is necessary to encourage students to explore computer science in college or beyond. But even for students whose futures eschew computing and technology, the benefits of computer science exposure and computational mindsets transcends scientific disciplines. \textbf{No matter what field of study students choose to persue, a solid foundation in computer science is one of the most powerful, flexible, creative, and innovative toolboxes for understanding and exploring our world in the 21st century.} \par




