% Appendix D

\chapter{Appendix: NGSS Standards} % Main appendix title

\label{AppendixNGSS} % For referencing this appendix elsewhere, use \ref{AppendixA}

\lhead{Appendix: \emph{NGSS Standards}}

\begin{longtable}{p{1.8cm}p{11cm}}
\caption{K-12 Engineering Design NGSS Standards~\cite{ngss}} \\\hline
\textbf{ID} & \textbf{DESCRIPTION} \\ \hline
K-2-ETS1-1. & Ask questions, make observations, and gather information about a situation people want to change to define a simple problem that can be solved through the development of a new or improved object or tool.\\ \hline
K-2-ETS1-2. & Develop a simple sketch, drawing, or physical model to illustrate how the shape of an object helps it function as needed to solve a given problem.\\ \hline
K-2-ETS1-3. & Analyze data from tests of two objects designed to solve the same problem to compare the strengths and weaknesses of how each performs.\\ \hline
3-5-ETS1-1. & Define a simple design problem reflecting a need or a want that includes specified criteria for success and constraints on materials, time, or cost.\\ \hline
3-5-ETS1-2. & Generate and compare multiple possible solutions to a problem based on how well each is likely to meet the criteria and constraints of the problem.\\ \hline
3-5-ETS1-3. & Plan and carry out fair tests in which variables are controlled and failure points are considered to identify aspects of a model or prototype that can be improved.\\ \hline
MS-ETS1-1. & Define the criteria and constraints of a design problem with sufficient precision to ensure a successful solution, taking into account relevant scientific principles and potential impacts on people and the natural environment that may limit possible solutions.\\ \hline
MS-ETS1-2. & Evaluate competing design solutions using a systematic process to determine how well they meet the criteria and constraints of the problem.\\ \hline
MS-ETS1-3. & Analyze data from tests to determine similarities and differences among several design solutions to identify the best characteristics of each that can be combined into a new solution to better meet the criteria for success.\\ \hline
MS-ETS1-4. & Develop a model to generate data for iterative testing and modification of a proposed object, tool, or process such that an optimal design can be achieved.\\ \hline
HS-ETS1-1. & Analyze a major global challenge to specify qualitative and quantitative criteria and constraints for solutions that account for societal needs and wants.\\ \hline
HS-ETS1-2. & Design a solution to a complex real-world problem by breaking it down into smaller, more manageable problems that can be solved through engineering.\\ \hline
HS-ETS1-3. & Evaluate a solution to a complex real-world problem based on prioritized criteria and trade-offs that account for a range of constraints, including cost, safety, reliability, and aesthetics as well as possible social, cultural, and environmental impacts.\\ \hline
HS-ETS1-4. & Use a computer simulation to model the impact of proposed solutions to a complex real-world problem with numerous criteria and constraints on interactions within and between systems relevant to the problem.\\ \hline
\end{longtable}		