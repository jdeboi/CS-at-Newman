\chapter{Appendix: Lower School Computer Brainstorm} % Main appendix title

\label{AppendixBRAIN} % For referencing this appendix elsewhere, use \ref{AppendixA}

\lhead{Appendix: \emph{Lower School Computer Brainstorm}} % This is for the header on each page - perhaps a shortened title

A facilitated brainstorm during an afternoon faculty meeting helped to elicit thoughts and concerns from all Lower School teachers regarding both computer literacy and coding. Faculty divided into groups based on a grade and answered the following questions on Post-its: \par
\begin{description}
	\item [Q1.] How is technology used in your classroom?
\item [Q2.] What do you wish/ think your students should know that they currently don't about coding and computer literacy?
\item [Q3.] What are your hopes and dreams for computing? 
\item [Q4.] What are your concerns about implementing coding curriculum at Newman?
\end{description}
The number of times particular subjects came up in responses to these questions is recorded in Table \ref{brainsum}.\par

\begin{longtable}{p{3cm}p{6cm}} 
	
\caption{LS Computer Brainstorm Summary} \\
\label{brainsum}
\text{Times Recurring} & \text{Subject} \\\hline
14 & professional development  \\
14 & word processing \\
12 & internet research \\
12 & safety and ethics \\
7 & saving \\
6 & keyboarding\\
6 & time restrictions\\
5 & balance between CS and coding\\
5 & email\\
5 & iPad \\
5 & troubleshooting\\
4 & images and video\\
4 & PowerPoint\\
3 & appropriate\\
3 & college prep\\
3 & creativity\\
3 & printing\\
2 & confidence\\
2 & CS in context\\
2 & curriculum development\\
2 & integrate into existing curriculum\\
1 & network\\
\end{longtable}


%\begin{longtable}{p{2cm}p{2.2cm}p{7cm}}
%\caption{LS Computer Brainstorm} \\
%\label{longbrain}
%\textbf{GRADE} & \textbf{QUESTION} & \textbf{RESPONSE} \\\hline
%PK & Q1 & no passwords \\
%PK & Q1 & Starfall \\
%PK & Q1 & Traffic Jam \\
%PK & Q1 & Kodable \\
%PK & Q1 & logic games - Gelato, Traffic Jam \\
%PK & Q1 & apps don't load \\
%PK & Q1 & Are iPads more PreK approp?  \\
%PK & Q1 & Handwriting w/o Tears \\
%PK & Q1 & Are comps PreK approp? \\
%PK & Q1 & We didn't choose apps \\
%PK & Q1 & Alien Buddies \\
%PK & Q1 & Letter Hunt \\
%PK & Q1 & Taking Picture Art/drawing apps \\
%K & Q1 & Students do not use a laptop \\
% K & Q1 & very limited use in K (by choice) \\
% K & Q1 & Handwriting without tears app - very rarely as follow up \\
% K & Q1 & Love the Elmo! \\
% K & Q1 & limited choices of apps... only can choose free apps. We would like to use other apps \\
% 1 & Q1 & Teach Me 1st app \\
% 1 & Q1 & Brain Pop Jr \\
% 1 & Q1 & The Weather App daily \\
% 1 & Q1 & Spelling City app on the iPad \\
% 1 & Q1 & Book Creator app \\
% 1 & Q1 & Only have iPads \\
% 1 & Q1 & The Camera \\
% 1 & Q1 & PicCollage app \\
% 1 & Q1 & Kodable and Scratch during media \\
% 1 & Q1 & Draw and Tell app \\
% 1 & Q1 & Word Bingo \\
% 2 & Q1 & word study - Spelling City, Drawing Pad \\
% 2 & Q1 & Research - All about books, culture project \\
% 2 & Q1 & reading - voice memos for fluency \\
% 2 & Q1 & Oregon Train \\
% 2 & Q1 & Videos - Discovery Ed, Brainpop Jr. \\
% 2 & Q1 & Document camera \\
% 2 & Q1 & Using less this year than last year \\
% 3 & Q1 & fast facts - iPads \\
% 3 & Q1 & laptop - ``Skype'' mystery another school \\
% 3 & Q1 & laptop - research \\
% 3 & Q1 & Scratch \\
% 3 & Q1 & multiplication games \\
% 3 & Q1 & iPad research \\
% 3 & Q1 & Typing Pal \\
% 3 & Q1 & Google docs \\
% 3 & Q1 & ``Wordle" \\
% 3 & Q1 & Google Drive (writing) \\
% 3 & Q1 & iPads - record voice, take videos and pics \\
% 3 & Q1 & iPad - Pixie: publishing stories \\
% 4 & Q1 & math games apps \\
% 4 & Q1 & researching for social studies \\
% 4 & Q1 & coding \\
% 4 & Q1 & Google Drive (writing) \\
% 4 & Q1 & Google form \\
% 4 & Q1 & Books of interest \\
% 4 & Q1 & Typing Pal \\
% 5 & Q1 & Wiki - keep track of links they visit \\
% 5 & Q1 & Google Docs \\
% 5 & Q1 & games \\
% 5 & Q1 & research web for 3 yearly research projects \\
% 5 & Q1 & reference \\
% 5 & Q1 & watch videos \\
% 5 & Q1 & projects- Excel \\
% 5 & Q1 & class blog with info and links \\
% 5 & Q1 & watch YouTube videos (documentaries and poems) \\
% 5 & Q1 & checking Newman email and emailing work to self to do at home \\
% 5 & Q1 & Book review blog \\
% 5 & Q1 & PowerPoint presentation \\
% 5 & Q1 & Google custom search \\
% 5 & Q1 & Audacity recording presentations and poetry reading \\
% 5 & Q1 & maps 101 - geography games \\
% 5 & Q1 & becoming a better type \\
% 5 & Q1 & not enough for math :( \\
% 5 & Q1 & weebly.com to create a website \\
% 5 & Q1 & library catalog online \\
% 5 & Q1 & word processing for writing \\
% 5 & Q1 & searching for current events \\
% 5 & Q1 & iNewman share links and videos and discussion \\
% Specialist & Q1 & LS library: computers every day: searching catalog, teaching researching skills \\
% Specialist & Q1 & I made YouTube videos for kids to practice at home what they're learning in class \\
% Specialist & Q1 & listening in Spanish, spelling, playing games, videos, internet, Power Point, brochure, create crossword games \\
% Specialist & Q1 & Programming LEGO WeDo, digital temp. probes, research / internet search \\
% Specialist & Q1 & Quizlet and Google Earth on the smartboard \\
% Specialist & Q1 & two laptops for students to individually use kid-friendly, Garageband-esque music loop "composition" programs \\
% Specialist & Q1 & online video to introduce topics, story time from space (NASA), data collection for NFS \\
% Specialist & Q1 & professional email (Office), web search, webinar, reports \\
% Specialist & Q1 & minimal interactive smartboard stuff \\
% Specialist & Q1 & iPads used in math class to record the process- app Susie found! \\
% Specialist & Q1 & I use projector to show images of artist's works \\
% Specialist & Q1 & Kindergarten arts students use laptop to view images \\
% Specialist & Q1 & Kids use my computer to search images for reference \\
% Specialist & Q1 & Use Google docs quite a bit- spreadsheets for data collection \\
% PK & Q2 & Basic usage of iPad \\
% PK & Q2 & Don't expect much \\
% K & Q2 & Read, write, type \\
% K & Q2 & Watch/ listen to stories as words are highlighted below \\
% 1 & Q2 & more of a balance between computer literacy and coding \\
% 1 & Q2 & take a good picture on an iPad \\
% 1 & Q2 & basic internet search \\
% 1 & Q2 & record their voices on the iPad \\
% 1 & Q2 & Uploading a picture into an app \\
% 1 & Q2 & How to make an uppercase letter on keyboard \\
% 1 & Q2 & anything on a computer/laptop \\
% 1 & Q2 & login on a computer \\
% 2 & Q2 & better at iPads than computers \\
% 2 & Q2 & could use keyboard better \\
% 2 & Q2 & basic desktop skills - keyboard, mouse, web browser \\
% 2 & Q2 & troubleshooting \\
% 2 & Q2 & more literate with games/ entertainment \\
% 3 & Q2 & video creation \\
% 3 & Q2 & frames stop motion animation \\
% 3 & Q2 & download and import pictures into documents \\
% 3 & Q2 & PowerPoint \\
% 3 & Q2 & Save to H drive \\
% 3 & Q2 & troubleshooting \\
% 3 & Q2 & Microsoft Word \\
% 4 & Q2 & research independently \\
% 4 & Q2 & typing \\
% 4 & Q2 & saving an image \\
% 4 & Q2 & saving documents in various locations \\
% 4 & Q2 & word processing (spacing, centering, etc.) \\
% 4 & Q2 & changing font size \\
% 5 & Q2 & attach document to email \\
% 5 & Q2 & check Newman email at home \\
% 5 & Q2 & email use basics \\
% 5 & Q2 & effectively search the web, not just images \\
% 5 & Q2 & How to read the whole document of a web page (scroll down!) \\
% 5 & Q2 & checking the print queue \\
% 5 & Q2 & how to print effectively \\
% 5 & Q2 & not blaming computer when it disappears \\
% 5 & Q2 & saving effectively \\
% 5 & Q2 & problem solving \\
% 5 & Q2 & double spacing, Times New Roman, size 12 \\
% 5 & Q2 & comfort with MS Office \\
% 5 & Q2 & setup word document \\
% 5 & Q2 & cut + paste \\
% Specialist & Q2 & write and use email \\
% Specialist & Q2 & How to type a web address \\
% Specialist & Q2 & search \\
% Specialist & Q2 & 1st - operate Quizlet independently \\
% Specialist & Q2 & keyboarding \\
% Specialist & Q2 & do not know how to type \\
% Specialist & Q2 & Use H drive \\
% Specialist & Q2 & presentation tools \\
% Specialist & Q2 & print \\
% Specialist & Q2 & They don't know where to save and retrieve docs \\
% Specialist & Q2 & save \\
% Specialist & Q2 & insert accent marks in Word \\
% Specialist & Q2 & create a poster \\
% Specialist & Q2 & type word processing \\
% Specialist & Q2 & copy/paste \\
% Specialist & Q2 & print one image (and not 100) \\
% Specialist & Q2 & Pk/K - nothing \\
% PK & Q3 & Confident and comfortable taking risks \\
% K & Q3 & use technology in creative way for projects \\
% K & Q3 & Connecting to others around the world \\
% 1 & Q3 & comfortable with taking risks with technology \\
% 1 & Q3 & appropriate uses of technology \\
% 1 & Q3 & internet safety \\
% 1 & Q3 & broad range of skills \\
% 2 & Q3 & Integrate with curriculum better \\
% 2 & Q3 & More options - kids can get bored with the same apps/ programs \\
% 3 & Q3 & creating positive digital footprint \\
% 3 & Q3 & communicating responsibly \\
% 3 & Q3 & Hire a part time teacher for grades 5-8 \\
% 4 & Q3 & researching using credible sources \\
% 4 & Q3 & troubleshooting (more independently) \\
% 4 & Q3 & word processing \\
% 5 & Q3 & understanding the world of tech/ social media \\
% 5 & Q3 & see it as a positive and a negative \\
% 5 & Q3 & etiquette \\
% 5 & Q3 & keeping yourself safe \\
% 5 & Q3 & understanding that people read and can see everything you put online \\
% 5 & Q3 & learning more math through tech \\
% 5 & Q3 & use it as a tool to benefit your learning \\
% 5 & Q3 & We'd love digital cameras with video capability \\
% Specialist & Q3 & know what's necessary to flourish in college \\
% Specialist & Q3 & leave Newman knowing the avenues to develop an app if they had a good idea - not necessarily knowing how to go about it \\
% Specialist & Q3 & students know how to use technology for practical and academic and creative uses \\
% Specialist & Q3 & view computers as something more than playing games \\
% Specialist & Q3 & knowing how to write an email to a teacher appropriately \\
% Specialist & Q3 & that information on the web is to be taken with a grain of salt \\
% Specialist & Q3 & basic research skills and media literacy \\
% Specialist & Q3 & to have the ability to aptly navigate the gigantic store of information available on the web \\
% Specialist & Q3 & to discern between fact and lies \\
% Specialist & Q3 & I don't know enough to know what kids need \\
% Specialist & Q3 & vision would be to support teachers as well \\
% Specialist & Q3 & creating presentation \\
% Specialist & Q3 & basic presentation skills \\
% Specialist & Q3 & more internet safety \\
% Specialist & Q3 & have strong ethical values when using tech \\
% Specialist & Q3 & media - interface safely \\
% Specialist & Q3 & be a global citizen \\
% Specialist & Q3 & be nimble - know 5 ways to do one thing \\
% Specialist & Q3 & basic word processsing \\
% PK & Q4 & I don't know anything about it. How do I teach it? \\
% K & Q4 & coding is beneficial for planning and problem solving - for K this might look different \\
% K & Q4 & Not sure how to teach it \\
% 1 & Q4 & lack of other computer/ iPad and technology instruction \\
% 1 & Q4 & 1st graders need to learn more than one skill/ side of technology \\
% 1 & Q4 & That I'm going to have to teach it \\
% 1 & Q4 & I don't have the necessary skills/ experience \\
% 1 & Q4 & Not enough teaching/planning time already \\
% 2 & Q4 & Don't want it to be sole focus of tech curriculum \\
% 2 & Q4 & connecting what they are doing to why they are doing it \\
% 2 & Q4 & a few kids solve the levels for most of the class \\
% 2 & Q4 & learn how to use Minecraft for teaching \\
% 3 & Q4 & teacher training \\
% 3 & Q4 & inappropriate usage \\
% 4 & Q4 & Not sure why kids need to code \\
% 4 & Q4 & I would have to learn it myself (don't know anything about it) \\
% 4 & Q4 & Takes time away from other academics \\
% 5 & Q4 & developmentally - how does it fit in? \\
% 5 & Q4 & integrate it - don't use it as an add on \\
% 5 & Q4 & enough professional development \\
% 5 & Q4 & when will I learn it? \\
% 5 & Q4 & time \\
% 5 & Q4 & What will we lose - where will it fit \\
% Specialist & Q4 & coding is not all there is to computer literacy \\
% Specialist & Q4 & creative or mechanical? \\
% Specialist & Q4 & It will be used for ``fun" and not able to have deep understanding \\
% Specialist & Q4 & I don't know anything about it, so that makes me nervous. What if they ask me for help? \\
% Specialist & Q4 & That I can't do it. \\
% Specialist & Q4 & that I have not a clue \\
% Specialist & Q4 & kids need basic training and so does the faculty \\
% Specialist & Q4 & hacking for the wrong people \\
% Specialist & Q4 & scheduling- we will have to water down other subjects to fit this in \\
% Specialist & Q4 & It should not take away core disciplines but be blended into project based learning -> with teacher training \\
% Specialist & Q4 & no concerns- seems exciting \\
% \end{longtable}