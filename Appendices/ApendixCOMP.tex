\chapter{Appendix: Computing At School Benchmarks} % Main appendix title

\label{AppendixCOMP} % For referencing this appendix elsewhere, use \ref{AppendixA}

\lhead{Appendix: \emph{Computing At School Benchmarks}} % This is for the header on each page - perhaps a shortened title

Computing at School (CAS) is a community of computer science educators and industry professionals with over 150 regional hubs in the United Kingdom. This community is passionate about computer science education, and in 2012, CAS published a set of standards titled, ``Computer Science: a Curriculum for Schools'' \ref{computas}, which are enumerated below.\par 
CAS breaks the standards into five subject areas: algorithms, programs, data, computers, and communication \&internet. These benchmarks are broken into ``Key Stages'':
\begin{itemize}
	\item Key Stage 1 (K-2nd)
	\item Key Stage 2 (2nd-5th)
	\item Key Stage 3 (5th-9th)
	\item Key Stage 4 (9th-12th)
	\item Post K-12 education
\end{itemize}

\textbf{KEY STAGE 1}\\
ALGORITHMS
\begin{enumerate}
\item Algorithms are sets of instructions for achieving goals, made up of pre-defined steps
*the ‘how to’ part of a recipe for a cake+.
\item Algorithms can be represented in simple formats [storyboards and narrative text].
\item They can describe everyday activities and can be followed by humans and by
computers.
\item Computers need more precise instructions than humans do.
\item Steps can be repeated and some steps can be made up of smaller steps.
\end{enumerate}
PROGRAMS
\begin{enumerate}
\item Computers (understood here to include all devices controlled by a processor, thus
including programmable toys, phones, game consoles and PCs) are controlled by
sequences of instructions.
\item A computer program is like the narrative part of a story, and the computer’s job is to
do what the narrator says. Computers have no intelligence, and so follow the
narrator’s instructions blindly.
\item Particular tasks can be accomplished by creating a program for a computer. Some
computers allow their users to create their own programs.
\item Computers typically accept inputs, follow a stored sequence of instructions and
produce outputs.
\item Programs can include repeated instructions.
\end{enumerate}
DATA
\begin{enumerate}
\item Information can be stored and communicated in a variety of forms e.g. numbers,
text, sound, image, video.
\item Computers use binary switches (on/off) to store information.
\item Binary (yes/no) answers can directly provide useful information (e.g. present or
absent), and be used for decision.
\item Computers are electronic devices using stored sequences of instructions.
\item Computers typically accept input and produce outputs, with examples of each in the
context of PCs.
\item Many devices now contain computers
\end{enumerate}
COMMUNICATION AND THE INTERNET
\begin{enumerate}
\item That the World Wide Web contains a very large amount of information.
\item Web browser is a program used to use view pages.
\item Each website has a unique name.
\item Enter a website address to view a specific website and navigate between pages and
sites using the hyperlinks.
\end{enumerate} \\

\textbf{KEY STAGE 2}\\
ALGORITHMS
\begin{enumerate}
\item Algorithms can be represented symbolically [flowcharts] or using instructions in a
clearly defined language [turtle graphics].
\item Algorithms can include selection (if) and repetition (loops).
\item Algorithms may be decomposed into component parts (procedures), each of which
itself contains an algorithm.
\item Algorithms should be stated without ambiguity and care and precision are necessary
to avoid errors.
\item Algorithms are developed according to a plan and then tested. Algorithms are
corrected if they fail these tests.
\item It can be easier to plan, test and correct parts of an algorithm separately.
\end{enumerate}

PROGRAMS
\begin{enumerate}
\item A computer program is a sequence of instructions written to perform a specified task
with a computer.
\item The idea of a program as a sequence of statements written in a programming
language [Scratch]
\item One or more mechanisms for selecting which statement sequence will be executed,
based upon the value of some data item
\item One or more mechanisms for repeating the execution of a sequence of statements,
and using the value of some data item to control the number of times the sequence
is repeated
\item Programs can model and simulate environments to answer “What if” questions. 
\item Programs can be created using visual tools. Programs can work with different types
of data. They can use a variety of control structures [ selections and procedures].
\item Programs are unambiguous and that care and precision is necessary to avoid errors.
\item Programs are developed according to a plan and then tested. Programs are
corrected if they fail these tests.
\item The behaviour of a program should be planned.
\item A well-written program tells a reader the story of how it works, both in the code and
in human-readable comments
\item A web page is an HTML script that constructs the visual appearance. It is also the
carrier for other code that can be processed by the browser.
\item Computers can be programmed so they appear to respond ‘intelligently’ to certain
inputs.
\end{enumerate}

DATA
\begin{enumerate}
\item Similar information can be represented in multiple.
Introduction to binary representation [representing names, objects or ideas as
sequences of 0s and 1s].
\item The difference between constants and variables in programs.
\item Difference between data and information.
\item Structured data can be stored in tables with rows and columns. Data in tables can be
sorted. Tables can be searched to answer questions. Searches can use one or more
columns of the table.
\item Data may contain errors and that this affects the search results and decisions based
on the data. Errors may be reduced using verification and validation.
Personal information should be accurate, stored securely, used for limited purposes
and treated with respect.
\end{enumerate}

COMPUTERS
\begin{enumerate}
\item Computers are devices for executing programs.
\item Application software is a computer program designed to perform user tasks.
\item The operating system is a software that manages the relationship between the
application software and the hardware
\item Computers consist of a number of hardware components each with a specific role
[e.g. CPU, Memory, Hard disk, mouse, monitor].
\item Both the operating system and application software store data (e.g. in memory and a
file system)
\item The above applies to devices with embedded computers (e.g. digital cameras),
handheld technology (e.g. smart phones) and personal computers.
\item A variety of operating systems and application software is typically available for the
same hardware.
\item Users can prevent or fix problems that occur with computers (e.g. connecting
hardware, protection against viruses)
\item Social and ethical issues raised by the role of computers in our lives.
\end{enumerate}

COMMUNICATION AND THE INTERNET
\begin{enumerate}
\item The Internet is a collection of computers connected together sharing the same way
of communicating. The internet is not the web, and the web is not the internet.
\item These connections can be made using a range of technologies (e.g. network cables,
telephone lines, wifi, mobile signals, carrier pigeons)
\item The Internet supports multiple services (e.g. the Web, e-mail, VoIP)
\item The relationship between web servers, web browsers, websites and web pages.
\item The format of URLs.
\item The role of search engines in allowing users to find specific web pages and a basic
understanding of how results may be ranked.
\item Issues of safety and security from a technical perspective.
\end{enumerate}

\textbf{KEY STAGE 3}\\
ALGORITHMS
\begin{enumerate}
\item An algorithm is a sequence of precise steps to solve a given problem.
\item A single problem may be solved by several different algorithms.
\item The choice of an algorithm to solve a problem is driven by what is required of the
solution [such as code complexity, speed, amount of memory used, amount of data,
the data source and the outputs required].
\item The need for accuracy of both algorithm and data [difficulty of data verification;
garbage in, garbage out]

PROGRAMS
\begin{enumerate}
\item Programming is a problem-solving activity, and there are typically many different
programs that can solve the same problem.
\item Variables and assignment.
\item Programs can work with different types of data [integers, characters, strings].
\item The use of relational operators and logic to control which program statements are
executed, and in what order //
- Simple use of AND, OR and NOT
- How relational operators are affected by negation *e.g. NOT (a>b) = a≤b+.
\item Abstraction by using functions and procedures (definition and call), including:\\
- Functions and procedures with parameters.\\
- Programs with more than one call of a single procedure.
\item Documenting programs to explain how they work.
\item Understanding the difference between errors in program syntax and errors in
meaning. Finding and correcting both kinds of errors.
\end{enumerate}

DATA
\begin{enumerate}
\item Introduction to binary manipulation.
\item Representations of:\\
- Unsigned integers\\
- Text. [Key point: each character is represented by a bit pattern. Meaning is
by convention only. Examples: Morse code, ASCII.]\\
- Sounds [both involving analogue to digital conversion, e.g. WAV, and free of
such conversion, e.g. MIDI]\\
- Pictures [e.g. bitmap] and video.
\item Many different things may share the same representation, or “the meaning of a bit
pattern is in the eye of the beholder” *e.g. the same bits could be interpreted as a
BMP file or a spreadsheet file; an 8-bit value could be interpreted as a character or
as a number].
\item The things that we perceive in the human world are not the same as what computers
manipulate, and translation in both directions is required [e.g. how sound waves are
converted into an MP3 file, and vice versa]
\item There are many different ways of representing a single thing in a computer. [For
example, a song could be represented as:\\
- A scanned image of the musical score, held as pixels
- A MIDI file of the notes
- A WAV or MP3 file of a performance]
\item Different representations suit different purposes [e.g. searching, editing, size,
fidelity].
\end{enumerate}

COMPUTERS
\begin{enumerate}
\item Computers are devices for executing programs
\item Computers are general-purpose devices (can be made to do many different things)
\item Not every computer is obviously a computer (most electronic devices contain
computational devices)
\item Basic architecture: CPU, storage (e.g. hard disk, main memory), input/output (e.g.
mouse, keyboard)
\item Computers are very fast, and getting faster all the time (Moore’s law)
\item Computers can ‘pretend’ to do more than one thing at a time, by switching between
different things very quickly
\end{enumerate}

COMMUNICATION AND THE INTERNET
\begin{enumerate}
\item A network is a collection of computers working together
\item An end-to-end understanding of what happens when a user requests a web page in a
browser, including:\\
- Browser and server exchange messages over the network\\
- What is in the messages [http request, and HTML]\\
- The structure of a web page - HTML, style sheets, hyperlinking to resources\\
- What the server does [fetch the file and send it back]\\
- What the browser does [interpret the file, fetch others, and display the lot]\\
\item How data is transported on the Internet
- Packets and packet switching\\
- Simple protocols: an agreed language for two computers to talk to each
other. [Radio protocols “over”, “out”; ack/nack; ethernet protocol: first use
of shared medium, with backoff.] \\
\item How search engines work and how to search effectively. Advanced search queries
with Boolean operators.
\end{enumerate}

\textbf{KEY STAGE 4}\\
ALGORITHMS
\begin{enumerate}
\item The choice of an algorithm should be influenced by the data structure and data
values that need to be manipulated.
\item Familiarity with several key algorithms [sorting and searching].
\item The design of algorithms includes the ability to easily re-author, validate, test and
correct the resulting code.
\item Different algorithms may have different performance characteristics for the same
task. 
\end{enumerate}

PROGRAMS
\begin{enumerate}
\item Manipulation of logical expressions, e.g. truth tables and Boolean valued variables.
\item Two-dimensional arrays (and higher).
\item Programming in a low level language.
\item Procedures that call procedures, to multiple levels. [Building one abstraction on top
of another.]
\item Programs that read and write persistent data in files.
\item Programs are developed to meet a specification, and are corrected if they do not
meet the specification.
\item Documenting programs helps explain how they work.
\end{enumerate}

DATA
\begin{enumerate}
\item Hexadecimal
\item Two’s complement signed integers
\item String manipulation
\item Data compression; lossless and lossy compression algorithms (example JPEG)
\item Problems of using discrete binary representations:
- Quantization: digital representations cannot represent analogue signals with
complete accuracy [e.g. a grey-scale picture may have 16, or 256, or more
levels of grey, but always a finite number of discrete steps]
- Sampling frequency: digital representations cannot represent continuous
space or time [e.g. a picture is represented using pixels, more or fewer, but
never continuous]
- Representing fractional numbers
\end{enumerate}

COMPUTERS
\begin{enumerate}
\item Logic gates: AND/OR/NOT. Circuits that add. Flip-flops, registers (**).
\item Von Neumann architecture: CPU, memory, addressing, the fetch-execute cycle and
low-level instruction sets. Assembly code. [LittleMan]
\item Compilers and interpreters (what they are; not how to build them).
\item Operating systems (control which programs run, and provide the filing system) and
virtual machines.
\end{enumerate}

COMMUNICATION AND THE INTERNET
\begin{enumerate}
\item Client/server model.
\item MAC address, IP address, Domain Name service, cookies.
\item Some “real” protocol. *Example: use telnet to interact with an HTTP server.]
Routing
\item Deadlock and livelock
\item Redundancy and error correction
\item Encryption and security
\end{enumerate}


