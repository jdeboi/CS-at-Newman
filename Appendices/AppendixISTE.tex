\chapter{Appendix: ISTE Standards} % Main appendix title

\label{AppendixISTE} % For referencing this appendix elsewhere, use \ref{AppendixISTE}

\lhead{Appendix: \emph{ISTE Standards}}
The International Society for Technology in Education (ISTE) compiled the ``Computer Science Standards for Educators" \cite{iste}. Where applicable, the ISTE CS standards are linked to CSTA standards.
\begin{enumerate}
	\item Demonstrate knowledge of Computer Science content and model important principles and concepts. 	
	\begin{enumerate}
		\item Demonstrate knowledge of and proficiency in data representation and abstraction 					L2CT12, L3BCP02
		\begin{enumerate}
			\item Effectively use primitive data types 				L3ACT07
			\item Demonstrate an understanding of static and dynamic data structures				L3BCT06
			\item Effectively use, manipulate, and explain various external data stores: various types (text, images, sound, etc.), various locations (local, server, cloud), etc. 				L2CT07, L3ACP06, L3ACT07
			\item Effectively use modeling and simulation to solve real-world problems 				L3ACP04, L3ACT08, L3BCT08, L3BCT09
		\end{enumerate} 
		\item Effectively design, develop, and test algorithms
		\begin{enumerate}					
			\item Using a modern, high-level programming language, construct correctly functioning programs involving simple and structured data types; compound boolean expressions; and sequential, conditional, and iterative control structures 				L2CP05
			\item Design and test algorithms and programming solutions to problems in different contexts (textual, numeric, graphic, etc.) using advanced data structures 				L2CT01
			\item Analyze algorithms by considering complexity, efficiency, aesthetics, and correctness				L3BCT04
			\item Demonstrate knowledge of two or more programming paradigms				L3ACP07
			\item Effectively use two or more development environments 				L3BC01
			\item Demonstrate knowledge of varied software development models and project management strategies 				L3ACT02
		\end{enumerate}
		\item Demonstrate knowledge of digital devices, systems, and networks
		\begin{enumerate}				
			\item Demonstrate an understanding of data representation at the machine level 				L3BCT07
			\item Demonstrate an understanding of machinelevel components and related issues of complexity 				L3BCD02
			\item Demonstrate an understanding of operating systems and networking in a structured computer system 				L3ACD05
			\item Demonstrate an understanding of the operation of computer networks and mobile computing devices 				L3ACD01
		\end{enumerate}
		\item Demonstrate an understanding of the role computer science plays and its impact in the modern world 					L3BCI02
		\begin{enumerate}
			\item Demonstrate an understanding of the social, ethical, and legal issues and impacts of computing, and attendant responsibilities of computer scientists and users 				L3ACI04
			\item Analyze the contributions of computer science to current and future innovations in sciences, humanities, the arts, and commerce				L3BCI04
		\end{enumerate}
	\end{enumerate}
	\item Effective teaching and learning strategies
	\begin{enumerate} 						
							
		\item Plan and teach computer science lessons/units using effective and engaging practices and methodologies 				\begin{enumerate}	
			\item Select a variety of real-world computing problems and project-based methodologies that support active and authentic learning and provide opportunities for creative and innovative thinking and problem solving 
			\item Demonstrate the use of a variety of collaborative groupings in lesson plans/units and assessments 					\item Design activities that require students to effectively describe computing artifacts and communicate results using multiple forms of media 				
			\item Develop lessons and methods that engage and empower learners from diverse cultural and linguistic backgrounds 				
			\item Identify problematic concepts and constructs in computer science and appropriate strategies to address them 				
			\item Design and implement developmentally appropriate learning opportunities supporting the diverse needs of all learners 				
			\item Create and implement multiple forms of assessment and use resulting data to capture student learning, provide remediation, and shape classroom instruction
		\end{enumerate}	
	\end{enumerate}
	
	\item Effective, safe, \& ethical learning environments
	\begin{enumerate} 												
		\item Design environments that promote effective teaching and learning in computer science classrooms and online learning environments and promote digital citizenship 
		\begin{enumerate}
			\item Promote and model the safe and effective use of computer hardware, software, peripherals, and networks				
			\item Plan for equitable and accessible classroom, lab, and online environments that support effective and engaging learning
		\end{enumerate}	
	\end{enumerate}	
	\item Effective professional knowledge and skills
	\begin{enumerate}
		\item Participate in, promote, and model ongoing professional development and life-long learning relative to computer science and computer science education	
		\begin{enumerate}				
			\item Identify and participate in professional computer science and computer science education societies, organizations, and groups that provide professional growth opportunities and resources				
			\item Demonstrate knowledge of evolving social and research issues relating to computer science and computer science education 				
			\item Identify local, state, and national content and professional standards and requirements affecting the teaching of secondary computer science	
		\end{enumerate}		
	\end{enumerate}	
\end{enumerate}			